\documentclass[a4paper]{article}
\usepackage{group}
\title{Langevin MC}
\author{B. Han, T.M. Hodgson, M. Holden \& M.Puza}
\usepackage{palatino}
\begin{document}
	\maketitle 
	Aim? Sample efficiently from high dimensional distributions of the form \(\pi(x) = Z^{-1} e^{-U(x)}\) where \(Z\) is a normalising constant.\\
	How? Use Langevin Dynamics\\
	Wait, what LD is an SDE? Yup, use a variety of methods to simulate the dynamics of this SDE: ULA, MALA, LM, HOLA. Both tamed and coordinate-wise tamed!\\
	Why does this work? The invariant measure of the Langevin SDE is exactly what we need!\\
	What goes wrong with ULA? Good question, we can find that out to motivate introducing other methods!\\
	\section{Why Bother?}
	\section{Langevin Equation}
		\subsection{Discretisation: Euler-Maruyama/LM/HOLA}
		
	\section{MALA}
	
	\section{Taming: Coordinate and otherwise}
	
	\section{Applications to Large Datasets}
		\subsection{Stochastic Gradient Langevin Dynamics}
\end{document}