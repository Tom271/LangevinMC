% \documentclass[a4paper]{article}

% \usepackage[english]{babel}
% \usepackage[utf8]{inputenc}
% \usepackage{amsmath}
% \usepackage{graphicx}
% \usepackage[colorinlistoftodos]{todonotes}
% \usepackage{titling}
% \usepackage[margin=40mm]{geometry}
% \usepackage{tikz}
% \usepackage{lipsum}
% \usepackage{amsfonts}
% \usepackage{amsmath,amssymb}
% \usepackage{breqn}

% \newtheorem{theorem}{Theorem}


% \newtheorem{lemma}[theorem]{Lemma}
% \newtheorem{corollary}{Corollary}[theorem]



% \begin{document}
% \setlength{\droptitle}{-0.5in}


\normalsize
\textbf{Stochastic Gradient Langevin Dynamics}\\
In this section, we will closely follow\ \cite{pitfalls}



\subsection{Introduction}

Normally, samples in machine learning are of huge sample sizes, for which most MCMC algorithms are not designed to process. 
As a result of the computational cost, several new approaches were proposed recently, Stochastic Gradient Langevin Dynamics (SGLD) is a popular one. 
SGLD is based on the Langevin Monte Carlo (LMC)
LMC – a discretization of a continuous-time process, it requires to compute the gradient of the log-posterior at the current fit of the parameter and avoid the accept/reject step.
SGLD – use unbiased estimator of the gradient of log-posterior based on subsampling, suitable for samples of huge size. 


\subsection{Governing Equation}
Recall the following equations:
Euler discretization of the Langevin SDE:
$$\theta_{n+1} = \theta_n - h \nabla U(\theta_n)+\sqrt{2h} Z_{n+1}$$, where $h > 0$ is a constant step size and $(Z_n)_{n\geq1}$ is a sequence of i.i.d standard d - dimensional Guassian vectors. 
To reduce the costs of the algorithms, we will switch to SGLD, for which we will replace $\nabla U$ with an unbiased estimate $\nabla U_0+(\frac{N}{p}) \sum_{i\in S}\nabla U_i$, where S is a minibatch of \{1,...., N\} with replacement of size p. Our iterations were then updated as
$$\theta_{n+1} = \theta_{n}-h \Bigg (\nabla U_0(\theta_n)+\frac{N}{p}\sum_{i\in S_{n+1}}\nabla U_i(\theta_n)\Bigg )+\sqrt{2h}Z_{n+1}$$
Stochastic Gradient Descent (SGD) is characterised by the same recursion as SGLD without the Gaussian noise, (the last term):


$$\theta_{n+1} = \theta_{n}-h \Bigg (\nabla U_0(\theta_n)+\frac{N}{p}\sum_{i\in S_{n+1}}\nabla U_i(\theta_n)\Bigg)$$

\subsection{Analysis in Wasserstein distance}
\subsection{Definitions and Notations in Markov chain theory}
Recall the following definitions:
$\mathcal{P}_2(\mathbb{R}^d)$ the set of probablity measures with finite second momet.\\
$\mathcal{B}(\mathbb{R}^d)$ the Borel $\sigma$ - algebra of $\mathbb{R}^d$.\\
For $\lambda, \nu \in \mathcal{P}_2(\mathbb{R}^d)$, we define the Wasserstein distance by 

$$W_2(\lambda, \nu) =\inf_{\xi \in \Pi(\lambda, \nu)}(\int_{\mathbb{\mathbb{R}^d \times \mathbb{R}^d}}||\theta-\vartheta)||^2 \xi(d\theta, d\vartheta))^{\frac{1}{2}}$$
where, $\Pi(\lambda, \nu)$ is the set of probablity measures $\xi$ on $\mathcal{B}(\mathbb{R}^d)\otimes\mathcal{B}(\mathbb{R}^d)$ satisfying for all $A \in \mathcal{B}(\mathbb{R}^d), \xi(A \times \mathbb{R}^d)= \lambda(A)$ and $\xi (\mathbb{R}^d \times A) = \nu(A)$.\\
For any probablity measure $\lambda$ on $\mathcal{B}(\mathbb{R}^d)$, we define $\lambda R$ for all $A \in \mathcal{B}(\mathbb{R}^d)$ by $\lambda R(A) = \int_{\mathbb{R}^d}\lambda(d\theta)R(\theta, A)$.\\
For all $k\in \mathbb{N}*$, we define the Markov kernel $R^k$ recursively by $R^1 = R$ and for all $\theta \in \mathbb{R}^d$ and $A \in \mathcal{B}(\mathbb{R}^d)$, $R^{k+1}(\theta, A) = \int_{\mathbb{R}^d}  R^k(\theta, d\vartheta)R(\vartheta, A).$\\
A probablity measure $\bar{\pi}$ is invariant for R if $\bar{\pi}R = \bar{\pi}$.\\
Our algorithms LMC, SGLD, SGD and SGLDFP algorithms are homogeneous Markov chains with Markov kernels denoted $R_{LMC}, R_{SGLD}, R_{SGD}$ and $R_{FP}$.

\subsection{Results}
For lemma 1, Theorem 2 and Corollary 3, we assume H1, H2 and H3.
\begin{lemma}
For any step size $h \in (0, \frac{2}{L})$, $R_{SGLD}$(respectively $R_{LMC}, R_{SGD}, R_{FP}$) has a unique invariant measure $\pi_{SGLD}\in \mathcal{P}_2(\mathbb{R}^d)$(respectively $\pi_{LMC}, \pi_{SGD}, \pi_{FP}$). In addition, for all $h \in (0, \frac{1}{L}], \theta\in \mathbb{R}^d and k\in\mathbb{N}$,
$$W_2^2(R_{SGLD}^k(\theta, \cdot), \pi_{SGLD})\leq(1-mh)^k\int_{\mathbb{R}^d}||\theta-\vartheta||^2\pi_{SGLD}(d\vartheta)$$
same inequality holds for LMC, SGD and SGLDFP.
\end{lemma}

\begin{theorem}
For all $h\in(0,\frac{1}{L}], \lambda, \nu\in \mathcal{P}_2(\mathbb{R}^d)\ and \ n\in\mathbb{N}$, we have the following upper- bounds in Wasserstein distance between
\begin{enumerate}
	\item 
	LMC and SGLDFP,

\begin{dmath}	
W_2^2(\lambda R_{LMC}^n, \nu R_{FP}^n)\leq(1-mh)^nW_2^2(\lambda, \nu) + \frac{2L^2h d}{pm^2}+\frac{L^2h^2}{p}n(1-mh)^{n-1}\int_{\mathbb{R}^d}||\vartheta-\theta*||^2 \mu(d\vartheta)
\end{dmath},
	\item 
	the Langevin diffusion and LMC,
\begin{dmath}
W_2^2(\lambda R_{LMC}^n, \mu P_{nh})\leq2(1-\frac{mLh}{m+L})^nW_2^2(\lambda, \mu)+dh\frac{m+L}{2m}(3+\frac{L}{m})(\frac{13}{6}+\frac{L}{m})\\+ne^{-(\frac{m}{2})h(n-1)}L^3h^3(1+\frac{m+L}{2m})\int_{\mathbb{R}^d}||\vartheta - \theta*||^2 \mu(d\vartheta)
\end{dmath},
	\item 
	SGLD and SGD
	\begin{dmath}
	W_2^2(\lambda R_{SGLD}^n, \mu R_{SGD}^n)\leq (1-mh)^n W_2^2(\lambda, \mu)+\frac{(2d)}{m}.
	\end{dmath}
\end{enumerate}
\end{theorem}
Proof omitted.
\begin{cor}
Set $h - \frac{\eta}{N} \ with \ \eta \in (0, \frac{1}{(2L)}]$ and assume that $lim \inf_{N \to \infty}mN^{-1}>0$. Then
\begin{enumerate}
\item
for all $n \in N$, we get $W_2(R_{LMC}^n(\theta*, \cdot), R_{FP}^{n}(\theta*, \cdot)) = \sqrt{d\eta}\mathcal{O}(N^{-\frac{1}{2}})$ and $W_2(\pi_{LMC}, \pi_{FP}) = \sqrt{d\eta}\mathcal{O}(N^{-\frac{1}{2}})$.
\item 
for all $n\in \mathbb{N}, we get W_2(R_{SGLD}^{n}(\theta*, \cdot), R^n_{SGD}(\theta*, \cdot)) = \sqrt{d}\mathcal{O}(N^{-\frac{1}{2}})$, and $W_2(\pi_{SGLD}, \pi_{SGD}) = \sqrt{d}\mathcal{O}(N^{-\frac{1}{2}})$.
\end{enumerate}
\end{cor}