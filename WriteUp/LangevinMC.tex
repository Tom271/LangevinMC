\documentclass[a4paper, titlepage, 11pt]{article}
\usepackage{amsmath}
\usepackage{amsthm}
\usepackage{amssymb}
\usepackage{amsfonts}
\usepackage{graphicx}
\usepackage[left=3cm,right=3cm,bottom=3.5cm,top=3.5cm]{geometry}
\usepackage{fancyhdr}
\usepackage{subcaption}
\usepackage{hyperref}
\usepackage{enumitem}
\usepackage{float}
\usepackage{breqn}
\setlength{\headheight}{14.5pt}

% Theorem-like environments
	\let\oldref\ref
	\renewcommand{\ref}[1]{(\oldref{#1})}
	\newtheorem{theorem}{Theorem}[section]
	\newtheorem{cor}[theorem]{Corollary}
	\newtheorem{lemma}[theorem]{Lemma}
	\newtheorem{prop}[theorem]{Proposition}
	\newtheorem{remark}[theorem]{Remark}
	\theoremstyle{definition}
	\newtheorem*{note}{Note}
	\newtheorem{defn}[theorem]{Definition}
	\newtheorem{ex}[theorem]{Example}

	\newenvironment{solution}{\renewcommand\qedsymbol{$\square$}\begin{proof}[Solution]}{\end{proof}}


% Set header and footer
%	\pagestyle{fancy}
%	\fancyhf{}

% I like footnotes to be superscript letters rather than numbers
	\renewcommand{\thefootnote}{\alph{footnote}}
	\setlist[enumerate,1]{label={\roman*)}}

%Swap bullets for long dash in itemize
	\renewcommand\labelitemi{---}
	\renewcommand\qedsymbol{\(\blacksquare\)}

% Typesets derivatives nicely using roman d. Adapts size automatically. Stolen from https://tex.stackexchange.com/questions/135944/commath-and-ifinner/135985#135985
% \od{y}{x} is ordinary derivative, \pd{f}{x} is partial derivative. See commath documentation for others
	\usepackage{amsmath}
	\newcommand{\dif}{\mathop{}\!\mathrm{d}}
	\newcommand{\Dif}{\mathop{}\!\mathrm{D}}

	\makeatletter
	\newcommand{\spx}[1]{%
		\if\relax\detokenize{#1}\relax
		\expandafter\@gobble
		\else
		\expandafter\@firstofone
		\fi
		{^{#1}}%
	}
	\makeatother

	\newcommand\pd[3][]{\frac{\partial\spx{#1}#2}{\partial#3\spx{#1}}}
	\newcommand\tpd[3][]{\tfrac{\partial\spx{#1}#2}{\partial#3\spx{#1}}}
	\newcommand\dpd[3][]{\dfrac{\partial\spx{#1}#2}{\partial#3\spx{#1}}}

	\newcommand{\md}[6]{\frac{\partial\spx{#2}#1}{\partial#3\spx{#4}\partial#5\spx{#6}}}
	\newcommand{\tmd}[6]{\tfrac{\partial\spx{#2}#1}{\partial#3\spx{#4}\partial#5\spx{#6}}}
	\newcommand{\dmd}[6]{\dfrac{\partial\spx{#2}#1}{\partial#3\spx{#4}\partial#5\spx{#6}}}

	\newcommand{\od}[3][]{\frac{\dif\spx{#1}#2}{\dif#3\spx{#1}}}
	\newcommand{\tod}[3][]{\tfrac{\dif\spx{#1}#2}{\dif#3\spx{#1}}}
	\newcommand{\dod}[3][]{\dfrac{\dif\spx{#1}#2}{\dif#3\spx{#1}}}

	\newcommand{\genericdel}[4]{%
		\ifcase#3\relax
		\ifx#1.\else#1\fi#4\ifx#2.\else#2\fi\or
		\bigl#1#4\bigr#2\or
		\Bigl#1#4\Bigr#2\or
		\biggl#1#4\biggr#2\or
		\Biggl#1#4\Biggr#2\else
		\left#1#4\right#2\fi
	}
	\newcommand{\del}[2][-1]{\genericdel(){#1}{#2}}
	\newcommand{\set}[2][-1]{\genericdel\{\}{#1}{#2}}
	\let\cbr\set
	\newcommand{\sbr}[2][-1]{\genericdel[]{#1}{#2}}
	\let\intoo\del
	\let\intcc\sbr
	\newcommand{\intoc}[2][-1]{\genericdel(]{#1}{#2}}
	\newcommand{\intco}[2][-1]{\genericdel[){#1}{#2}}
	\newcommand{\eval}[2][-1]{\genericdel.|{#1}{#2}}
	\newcommand{\envert}[2][-1]{\genericdel||{#1}{#2}}
	\let\abs\envert
	\newcommand{\sVert}[1][0]{%
		\ifcase#1\relax
		\rvert\or\bigr|\or\Bigr|\or\biggr|\or\Biggr
		\fi
	}
	\newcommand{\enVert}[2][-1]{\genericdel\|\|{#1}{#2}}
	\let\norm\enVert
	\newcommand{\fullfunction}[5]{%
		\begin{array}{@{}r@{}r@{}c@{}l@{}}
			#1 \colon & #2 & {}\longrightarrow{} & #3 \\
			& #4 & {}\longmapsto{}     & #5
		\end{array}
	}


\title{Langevin Monte Carlo \\
  \large and the curse of dimensionality}
\author{B. Han, T.M. Hodgson, M. Holden \& M. Puza\\supervised by\\Dr Sotirios Sabanis}

% Define new commands for typesetting here
\newcommand{\R}{\mathbb{R}}
\renewcommand{\L}{\mathcal{L}}
\renewcommand{\P}{\mathbb{P}}
%\renewcommand{\epsilon}{\varepsilon}
\newcommand{\e}{\mathrm{e}}
\newcommand{\grad}{\nabla}
\newcommand{\E}{\mathbb{E}}



\begin{document}
	\maketitle
	\section*{Abstract}
	Monte Carlo methods are introduced, with a focus on Langevin-based samplers (Section \ref{sec:LMC}). The method of taming is studied and compared to Metropolised algorithms using a variety of metrics (Sections \ref{sec:MomentErrors},\ref{sec:Beyond}). Our main contribution is an extension of the paper by Brosse et al. (2018) on the tamed unadjusted Langevin algorithm, via an open source \textsc{Python} package for comparison of a variety of Langevin Monte Carlo (LMC) algorithms on a range of distributions. We have also extended the visualisation library of \cite{rogozhnikov2016hmc} to include LMC algorithms. This is presented in Section \ref{sec:Imp}, along with a discussion of the implemented metrics and pitfalls.
	\section*{Author Contribution}
	Sections \ref{sec:intro} and \ref{sec:Beyond} were written by Matthew.  Tom wrote \ref{sec:LMC}, \ref{sec:MomentErrors} and \ref{subsec:future}.  Marko wrote 5.2
	- 5.5. Bowen wrote \ref{overviewprogram} and \ref{subsec:SGLD}.\\
	The \textsc{Python} package available on GitHub was contributed to by Tom, Matthew and Marko, and is primarily the work of Marko.
	\tableofcontents
	\newpage
	\section{Introduction}\label{sec:intro}
		\section{Introduction}
	Markov chain Monte Carlo (MCMC) methods are a family of algorithms for calculating numerical approximations of integrals, with applications in computational physics, biology and statistics.  
	
	\subsection*{Monte Carlo Integration}
	Th
		
	
	



	
	\section{Langevin Monte Carlo Algorithms}\label{sec:LMC}
	For the remainder, we will focus on distributions of the form 
\[ \pi(x)=\mathcal{Z}^{-1}\e^{-U(x)},\]
where \(\mathcal{Z}\) is a normalising constant. Borrowing some terminology from physics, we call the function \(U:\R^d\to\R\) the potential function. In statistical mechanics, this distribution is known as the Boltzmann distribution. The aim here is to use machinery from physics to solve the problem of sampling accurately. To do this we first must motivate the distribution's origin.

Consider a particle in a potential well of shape \(U\). The equation of motion for such a particle is \cite{Langevin1908}, 
\begin{equation} \dif X_t = -\nabla U(X_t)\dif t +\sqrt{2}\dif W_t. \label{eq:ODLang}.\end{equation}
Here, \(X_t\) is the displacement of the particle from the origin at time \(t\), \(W_t\) is a \(d\)-dimensional Wiener process (Brownian motion) and \(U:\R^d \to \R\) is the potential function. The form of interest here is the \emph{overdamped} Langevin equation, in which the particle experiences no average acceleration, obtained as the high friction limit of full Langevin dynamics.

From the form of the equation, it can be seen that the particle moves down the slope of the well, ``sliding'' towards the minimum energy state. As each particle moves randomly, it is natural to ask what is the average position of many particles in such a well? One would expect the average position to be the one that has lowest energy. To recast the question in statistical language, we ask what is the expected value of the random variable \(X_t\)? This is exactly the goal of MCMC given in Section \ref{sec:intro}. In fact for the Langevin equation, we can give the probability of a particle being at any given displacement. This probability is given exactly by the measure \(\pi\), the Boltzmann distribution. For a diffusion process such as Equation \ref{eq:ODLang} this is called the \emph{stationary distribution}\footnote{Another common term is \emph{invariant measure}}. To show that \(\pi\) is indeed the stationary distribution we use the following lemma.

\begin{lemma}
	For a one-dimensional It\^o diffusion\footnote{That is \(X_t\) solves \(\dif X_t = \mu(X_t)\dif t +\sigma^2(X_t)\dif W_t\)}, suppose \(\sigma^2(t,x)\) is bounded. Suppose that the transition probabilities of \(X_t\) admit a density for every \(t>0\). Further assume that \(\mu(X_t),\sigma^2(X_t) \in C^2\). Define the Fokker-Planck operator, \(\L^*\), as
	\[\L^*:= -\partial_x(\mu(x)\cdot)+\frac{1}{2}\partial^2_x(\sigma^2(x)\cdot).\]
	Then a measure \(\pi\) is invariant for the diffusion if and only if
	\[\L^*\pi = 0\]
\end{lemma}
The proof of this is omitted however it can be seen by forming the Fokker-Planck equation for the probability density of the diffusion. If a diffusion admits a unique stationary measure, then it is ergodic. This fact means one can apply Theorem \eqref{thm:ergodic} and calculate expectations using Langevin dynamics. The proof that \(\pi\) is the stationary measure of Equation \eqref{eq:ODLang} is given only in the one dimensional case, however it is extendable to higher dimensions. For the Langevin equation, the Fokker-Planck operator is

\[\L^* = \partial_x(U'(x)\cdot)+\partial_{xx}\cdot . \]
So it remains to calculate \(\L^*\pi\).
\begin{align*}
\L^*\pi &= \pd{}{x}\bigg\lbrack U'(x)\pi(x) + \pd{}{x}\pi(x)\bigg\rbrack\\
		&= \pd{}{x}\bigg\lbrack U'(x)\mathcal{Z}\e^{-U(x)}+ \left(-U'(x)\mathcal{Z}\e^{-U(x)}\right)\bigg\rbrack\\
		&= \pd{}{x}\lbrack 0 \rbrack\\
		&= 0
\end{align*}
Hence \(\pi\) is indeed the invariant measure of \eqref{eq:ODLang}. \qed
\\
\\
Although this shows that the Langevin equation has an invariant measure, the question of convergence to this measure remains unanswered. Roberts and Tweedie give the following restriction \cite{RT96}.
\begin{theorem}[Theorem 2.1, \cite{RT96}]
	Let \(P^t_X(x,A) = \P(X_t\in A | X_0 =x_0)\) and suppose that \(\grad U(x)\) is continuously differentiable and that, for some \(N,a,b < \infty\),
	\[\grad U(x)\cdot x \leq a|x|^2 + b, \qquad |x|>N. \]
	Then the measure \(\pi\) is invariant for the Langevin diffusion \(X\). Moreover, for all \(x \in \R^d \) and Borel sets \(A\),
	\[\|P^t_X(x,\cdot) - \pi \| = \frac{1}{2}\sup_A \big|P^t_X(x,A)-\pi(A)\big| \to 0\]

\end{theorem}

\begin{figure}[ht]
	\centering
		\includegraphics[width=\linewidth]{quadraticLD.pdf}
	\caption{Simulating Langevin dynamics in one dimension with a quadratic potential \(U(x)=x^2/2\)}
	\label{fig:quadLD}
\end{figure}

The problem of sampling from the high dimensional distribution has been reduced to being able to accurately simulate Langevin dynamics. This is not as simple as it sounds. To simulate the continuous process  \eqref{eq:ODLang}, it must first be discretised. Doing so may not preserve the convergence to the invariant measure. The discretised process may not have the same stationary measure or it may not even exist. This means that the method used to discretise must be chosen carefully to ensure good convergence properties. The most natural way to discretise an SDE is to use the stochastic analogue of the (forward) Euler method used on ordinary differential equations, known as the Euler-Maruyama (EM) method. Doing so leads to the Unadjusted Langevin Algorithm (\texttt{ULA}).

\subsection{The Unadjusted Langevin Algorithm}
Applying the Euler-Maruyama method to Equation \eqref{eq:ODLang} gives the following iterative scheme.

\[X_{n+1} = X_n -h \nabla U(X_n) +\sqrt{2h} Z_{n+1},\qquad X_0= x_0 \]
Here the \(Z_n \) are i.i.d. standard normal random variables and \(h\) is the step size. This is equivalent to \(X_{n+1} \sim N(X_n - h\grad U(X_n), 2h I_d )\).\footnote{\(I_d\) denotes the \(d \times d\) identity matrix.} A simple example shows that this discretisation does not converge to \(\pi\). Let \(\pi\) be a standard Gaussian distribution, that is \(U(x) = |x|^2/2 \) and choose \(h = 1\). Then the update is given by

\begin{align*}
	X_{n+1} &\sim N(X_n - \grad U(X_n), 2)\\
	& \sim  N(X_n - X_n, 2)\\
	& \sim N(0,2) \nsim \pi .
\end{align*}
So the chain converges immediately, but to the wrong distribution. Let \(\pi^{\text{ULA}}_{h} \) denote the stationary distribution of \texttt{ULA} with a stepsize \(h\). This is not the only issue that can occur. As well as not converging to the correct distribution, the discretised chain may not be  ergodic, even when the continuous diffusion is exponentially ergodic \cite{RT96}. In particular, the algorithm misbehaves when the gradient of the potential is superlinear. That is,
\[\liminf_{\|x\|\to \infty} \frac{\|\grad U(x)\|}{\|x\|} = +\infty. \]
To mitigate these issues there are two main approaches: taming the gradient and Metropolisation. A further third method involves using a different discretisation scheme.  Our main focus will be the former, although all three approaches will be discussed.

\subsection{Metropolis Adjustment}
Before describing the Metropolis-adjusted Langevin algorithm \texttt{MALA}, it is pertinent at this point to recall the random walk Metropolis-Hastings algorithm \texttt{RWM }\cite{Hastings70, Metropolis53}. This popular variant of the Metropolis-Hastings algorithm \emph{proposes} values and then accepts/rejects them according to some probability \(\alpha\).  So given \(X_n\), propose a candidate \(Y_{n+1}\) as

\[Y_{n+1} = X_n  + \sqrt{2h} Z_{n+1}.\]
Once again, \(h\) is the stepsize and \(Z\) is a normal random variable. Then, accept or reject this proposal using Metropolis rejection, that is with some probability
\[\alpha(X_n,Y_{n+1}) = 1\wedge \frac{\pi(Y_{n+1})q(Y_{n+1},X_n)}{\pi(X_n)q(X_n,Y_{n+1})}.\footnote{Here \(t\wedge s = \min\lbrace t,s\rbrace.\) }\]
Here \(q(x,y)\) is the transition probability, \(\P(Y_{n+1}=y | X_{n}=x)\sim N(X_n, h^2)\). This rejection step is key in creating a kernel that is reversible and thus invariant for the measure \(\pi\). \\


\texttt{MALA} can be seen as another variant of the Metropolis-Hastings algorithm, using Langevin dynamics to propose new states as follows.
\[Y_{n+1} = X_n -h\grad U(X_n) + \sqrt{2h} Z_{n+1}\]
It is perhaps better understood as \texttt{ULA} but with an added Metropolis rejection step \cite{RT96}. Adding this rejection step means the algorithm always has the correct invariant distribution, although convergence is still not guaranteed as the following theorem shows.

\begin{theorem}[Theroem 4.2, \cite{RT96}]
	If \(\pi\) is bounded, and
		\[\liminf_{\|x\|\to \infty} \frac{\|\grad U(x)\|}{\|x\|} > \frac{4}{h}\]
	then the \texttt{MALA} chain is not exponentially ergodic. +++define exp ergodic+++
\end{theorem}
So it can be seen that \texttt{MALA} is not without its issues, and does not solve all the problems of \texttt{ULA}. The concept of taming was introduced to try and reduce the magnitude of these problems.
\begin{figure}[H]
\centering
  \begin{minipage}[b]{0.49\textwidth}
  \centering
    \includegraphics[width=\textwidth]{Figures/tula_tmala_step_1.png}
  \end{minipage} %
  \begin{minipage}[b]{0.49\textwidth}
  \centering
    \includegraphics[width=\textwidth]{Figures/tula_tmala_step_10.png}
  \end{minipage}
   \caption{\textbf{Trade-off between rejection-based algorithms and \texttt{tULA} for large step-sizes} ($h = 1$ on the left and $h = 10$ on the right, distribution is Gaussian with covariance matrix $\text{diag}(1.0, 0.1)$). With increasing step size, even \texttt{tULA} starts to suffer from stiffness problems. Rejection-based algorithms resolve the issue, however, their acceptance rate drops very low ($\approx 0.2$ on the left and $\approx 0.03$ on the right). }
\end{figure}

\subsection{Taming the Gradient}
We have seen that both \texttt{ULA} and \texttt{MALA} run into issues when the gradient of the potential is superlinear. Given an SDE such as \eqref{eq:ODLang}, taming adjusts the drift coefficient in such a way that preserves the invariant measure and improves speed of convergence \cite{Brosse18tULA,RT96,Sabanis13}. To do this, a family of drift functions \((G_h)_{h>0}, \ G_h:\R^d \to \R^d\) are introduced. The SDE to be discretised is thus
	\begin{equation*} \dif X_t = -G_h(X_t)\dif t +\sqrt{2}\dif W_t. \end{equation*}
Applying the Euler-Maruyama method gives the following Markov chain
	\[X_{k+1} =X_k-hG_h(X_k)+\sqrt{2h}Z_{k+1},\qquad  X_0=x_0.\]
To preserve the invariant measure, some restrictions must be placed on \((G_h)_{h>0}\), namely that they are `close' to \(\grad U\) ({\bf A1}) while {\bf A2} ensures ergodicity is preserved and improves stability \cite{Brosse18tULA}. 

\begin{enumerate}[label={\bf A{\arabic*}}]
	\item  For all \(h>0, G_h\) is continuous. There exist \(\alpha\geq 0, C_{\alpha}<+\infty\) such that for all \(h >0 \) and \(x \in \R^d\),
		\[\|G_h(x)-\grad U(x)\| \leq hC_{\alpha}(1+\|x\|^{\alpha}).\]\label{A1}
	\item For all \(h>0\),
		\[ \liminf_{\|x\|\to \infty} \bigg\lbrack \bigg\langle \frac{x}{\|x\|}, G_h(x)\bigg\rangle - \frac{h}{2\|x\|}\|G_h(x)\|^2\bigg\rbrack >0\]\label{A2}
\end{enumerate}
Here we consider two specific taming functions,
 \begin{align*}
 T_h(x) = \frac{\grad U(x)}{1+h\|\grad U(x)\|}, &&  T^{\text{\sc \tiny RT}}_h = \frac{\grad U(x)}{1\vee h\|\grad U(x)\|}.
 \end{align*}
Brosse et al. introduced and studied \(T_h\) whilst Roberts \& Tweedie suggested \(T^{\text{\sc \tiny RT}}_h\), later analysed by Bou-Rabee \& Vanden-Eijnden \cite{BV10MALTA,Brosse18tULA,RT96}. Both taming functions retain the direction of the gradient, only reducing the magnitude of its effect. The latter is the usual \texttt{ULA} until the gradient gets large enough \((\|\grad U(x)\|> 1/h)\), at which point it begins normalising. In contrast, the first will always tame, regardless of size of the gradient. However for the scaling to have noticeable effect, the gradient must be \(\mathcal{O}(h^{-1})\).
When \(T_h\) is the taming function, the algorithm will be referred to as \texttt{tULA}, the tamed unadjusted Langevin algorithm. When the second is applied, it will be called \texttt{MALTA}, the Metropolis adjusted Langevin truncated algorithm after \cite{RT96}. Any tamed algorithm using \(T_h\) will be prefixed with a lowercase \texttt{t}. For a proof that \(T_h\) satisfies \ref{A1} and \ref{A2}, see \cite[Lemma~2]{Brosse18tULA}.
\\
When the problem is ill-conditioned, taming the gradient does not help. 
\subsubsection{tULA/c}
So far, the gradient has only been tamed globally. This means that the information the gradient gives is reduced in dimensions where it is not causing divergence. A solution to this is to use coordinatewise taming with the following drift.
  \[T^c_{h}(x) =\left(\frac{\partial_i U(x)}{1+h|\partial_i U(x)|}\right)_{i=\lbrace 1, \dots, d\rbrace} \]
This allows each dimension to be scaled individually. Any algorithm with coordinate-wise taming will be suffixed with a lowercase \texttt{c}.

\subsubsection{Stiff Problems}
To illustrate the effectiveness of coordinatewise taming, consider an ill-conditioned Gaussian distribution with mean \(\mu,\) covariance matrix \(\Sigma\) as follows.
\begin{align*}
    \mu &= \mathbf{0}, && \Sigma = \begin{bmatrix}  1 & 0 \\ 0 & 0.0001 \end{bmatrix}
\end{align*}
Figure \ref{fig:stiffULA} shows the trajectory of the unadjusted and tamed unadjusted Langevin algorithms (\texttt{ULA} and \texttt{tULA}, respectively) applied to this problem. It can be seen that past a certain threshold in stepsize, \texttt{ULA} begins to overestimate the width of the distribution, and indeed diverges when the stepsize gets too high. Notice that the tamed algorithm still overestimates the width in one dimension due to the global taming. Contrast this with Figure \ref{fig:stifftULAc}, using the \texttt{tULAc} algorithm. it greatly reduces the variance in samples in one dimension. The problem still remains in the other dimension, however this is almost unavoidable with an explicit scheme.  
\begin{figure}[H]
\centering
  \begin{minipage}[b]{0.49\textwidth}
  \centering
    \includegraphics[width=\textwidth]{Figures/ula_tula_step_01.png}
  \end{minipage} %
  \begin{minipage}[b]{0.49\textwidth}
  \centering
    \includegraphics[width=\textwidth]{Figures/ula_tula_step_02.png}
  \end{minipage}
   \caption{\textbf{Demonstration of the stiffness problem with \texttt{ULA}}, resolved by taming. Both \texttt{ULA} and \texttt{tULA} work well for step size $h = 0.1$ on the left, however \texttt{ULA} becomes stiff for a larger step size $h = 0.2$ on the right. For ever higher step sizes, \texttt{ULA} diverges. The distribution here is Gaussian with covariance matrix $\text{diag}(1.0, 0.0001)$.}
   \label{fig:stiffULA}
\end{figure}

\begin{figure}[H]
\centering
  \begin{minipage}[b]{0.49\textwidth}
  \centering
    \includegraphics[width=\textwidth]{Figures/tula_tulac_rwm_stiff.png}
  \end{minipage} %
   \caption{\textbf{Coordinate-wise taming} can deal with stiffness in one axis (Ill-conditioned Gaussian distribution with covariance $\text{diag}(1.0, 0.0001)$).}
   \label{fig:stifftULAc}
\end{figure}




\subsection{Discretise Differently}
An alternative approach is to use a different discretisation of the SDE \eqref{eq:ODLang}, which we consider in this section. The first is an extension of the Euler method \cite{Sabanis18tHOLA}, while the latter uses a non-Markovian scheme developed for use in molecular dynamics \cite{LM12}.
\subsubsection{Higher Order Langevin Algorithm}
As in the ordinary case, the Euler-Maruyama method is not the only way of discretising an SDE. One can also take a higher order expansion, analogous to the Runge-Kutta method in ODE theory, known as the order 1.5 Wagner-Platen expansion\footnote{Or the stochastic Runge-Kutta method \cite{Schaffter10numericalintegration}}. For a one dimensional Langevin diffusion \eqref{eq:ODLang}, this is 
\[X_{n+1} = X_n -hU'(X_n)+\sqrt{2h}Z_n -\sqrt{2} U''(X_n) \tilde{Z}_n +\frac{ h^2 }{2}\bigg\lbrack U'(X_n)U''(X_n)-U'''(X_n)\bigg\rbrack.  \]
Here, \(\tilde{Z}_n\) is defined as
\[  \tilde{Z}_n = \int_{t_n}^{t_{n+1}} \int_{t_n}^s \dif W_r \dif s. \]
This is a Gaussian random variable with mean \(0\) and variance \(\frac{1}{3}h^3 \). Extending this to \(d\)-dimensions and applying to the Langevin SDE \eqref{eq:ODLang} and taming as above gives the following iterative scheme. The untamed version is the same, but with all subscripts removed. 
\[X_{n+1} = X_n + \mu_{h}(X_n)h +\sigma_{h}(X_n)\sqrt{h}Z_{n+1},\]
where
\[\mu_{h}(x) = -\grad U_{h}(x) +\frac{h}{2}\left( \left( \grad^2U\grad U\right)_{h}(x) - \vec{\Delta}(\grad U)_{h}(x)\right) ,\]
and \(\sigma_{h}(x) = \text{diag}\left(\left( \sigma_{h}^{(k)}(x)\right)_{k\in \lbrace 1,\dots,d\rbrace}\right)\) with,
\[\sigma_{h}^{(k)}(x) = \sqrt{2+\frac{2h^2}{3}\sum_{j=1}^d |\grad^2 U_{h}^{(k,j)}(x)|^2 - 2h \grad^2 U_{h}^{(k,k)}(x)}.\]
The subscript \(h\) indicates a taming of the variable has occurred as follows. For any \(x\in \R^d\),
\begin{align*}
    \grad U_h(x) &=\frac{\grad U(x)}{(1+h^{3/2}|\grad U(x)|^{3/2})^{2/3}}\, , && \grad^2U_h(x) = \frac{\grad^2 U(x)}{1+h|\grad^2 U(x)|} ,\\
    (\grad U\grad^2U)_h(x) =&\frac{\grad^2U(x)\grad U(x)}{1+h|x||\grad^2U(x)||\grad U(x)|}\, , && \vec{\Delta}(\grad U)_h(x) = \frac{\vec{\Delta}(\grad U)(x)}{1+h^{1/2}|x||\vec{\Delta}(\grad U)(x)|}.
\end{align*}
Like in the classical case, the aim here is to improve the rate of convergence by using a more accurate discretisation of the underlying diffusion.

\subsubsection{Leimkuhler-Matthews Method}
The Leimkuhler-Matthews method was developed in \cite{LM12} and cleverly exploits the link between the sampling problem and molecular dynamics. The scheme they developed is as follows:

\begin{equation} X_{n+1} = X_n - h \grad U (X_n) +\sqrt{\frac{h}{2}} (Z_n+Z_{n+1}) \label{eq:LM} \end{equation}

Note that this method is non-Markovian as it incorporates the noise term from the previous iteration. At first glance it appears that this is almost identical to the standard EM scheme, however the slight modification drastically improves convergence.  It is derived by considering both positions and momenta in Langevin dynamics (as opposed to the overdamped equation we have considered thus far which neglects momenta).

\begin{align} \dif X_t = P_t \dif t && \dif P_t = \lbrack -\grad U(X_t) - \gamma P_t \rbrack \dif t + \sqrt{2} \dif W_t \label{eq:LMLang} \end{align}

Here, \(\gamma\) is the friction coefficient. It does no appear in the overdamped Langevin equation as it is the high friction limit of the above system.  Equation \eqref{eq:LMLang} can be seen as a combination of an Ornstein-Uhlenbeck process and Hamiltonian dynamics. Splitting the system in this way allows the exact solution of each part in each time step.

\[ \dif \begin{bmatrix} X_t\\P_t \end{bmatrix} = \underbrace{\begin{bmatrix} P_t \\ 0 \end{bmatrix}}_{\text{A}} \dif t + \underbrace{\begin{bmatrix}  0\\-\grad U \end{bmatrix}}_{\text{B}} \dif t + \underbrace{\begin{bmatrix} 0\\ -\gamma P_t \dif t + \dif W_t \end{bmatrix}}_{\text{O}} \]

The Hamiltonian dynamics have been split further in to A and B.  As well as solving these sections separately at each time step, it is also possible to do fractions of a time step for each term. If we solve in the order BAOAB, that is, half a time step each of B and A before a full time step of the OU process before solving the rest of A and B, we recover Equation \eqref{eq:LM} upon taking the high friction limit. Despite the dependence of iterates, the noise quickly decorrelates. Let \(R_n = (Z_n+Z_{n+1})/\sqrt{2}\). Then,

\begin{align*}
    \langle R_n,R_n\rangle  &= \frac{1}{2} \left( \langle Z_{n+1},Z_{n+1}\rangle +\langle Z_{n},Z_{n}\rangle\right) =1\\
    \langle R_n,R_n\rangle = \langle Z_n,Z_n\rangle = \frac{1}{2}\\
    \langle R_n,R_{n-k} \rangle &= 0,   \quad k=2,3,\dots 
\end{align*}

{\bf +++ LINK IN TO NEXT SECTION +++ }







\subsection{Visualization}
A demonstration of the above methods has been implemented using the visualization library of \cite{rogozhnikov2016hmc}\footnote{With kind permission of Alex Rogozhnikov, \url{https://arogozhnikov.github.io/about/}.}. The visualization dynamically follows the trace of a chosen method applied to a chosen two-dimensional distribution. Distributions of various qualitative properties are available. This can be found at the following \textsc{url}: \\
   \centerline{ \url{http://goatleaps.xyz/assets/ULA/ULA.html}}

\begin{figure}[H]
\centering
  \begin{minipage}[b]{0.8\textwidth}
  \centering
    \includegraphics[width=0.8\textwidth]{Figures/ulavis.PNG}
    \caption{Screenshot from the visualization;  tULAc applied to a Gaussian mixture distribution.}
  \end{minipage}
\end{figure}


    
    \section{Approximation Error}\label{sec:MomentErrors}
    Following the work of \cite{Brosse18tULA}, we aim to quantify the accuracy of these methods by finding the first and second moments of known distributions. Then, comparing the error between the generated value and the true value will give a first indication on the accuracy of the scheme. Beginning with the original code associated with \cite{Brosse18tULA}\footnote{Available at \url{https://github.com/nbrosse/TULA}}, we initially optimised the code then aimed to reproduce the results they found. 

\subsection{Testing Potentials}
For the purposes of testing, four qualitatively different potentials have been made available within our program. These are: Gaussian, Double Well, Rosenbrock function and Ginzburg-Landau model. We note that these, save for Gaussian, are non-convex.

Any of the above potentials may also be scaled by \textit{temperature} $T$. That is, having chosen a potential function $U$ and a temperature $T$, the true distribution to be sampled from will be
\[\pi(x) = e^{-\frac{U(x)}{T}}.\]
This is common in the molecular dynamics literature and is also useful in MCMC to find modes of distributions more quickly, a technique known as tempering.

+++ BoxPlots +++
\begin{figure}
\centering
  \begin{minipage}[b]{0.32\textwidth}
  \centering
    \includegraphics[width=\textwidth]{Figures/tula_fm.png}
  \end{minipage} %
  \begin{minipage}[b]{0.32\textwidth}
  \centering
    \includegraphics[width=\textwidth]{Figures/tulac_fm.png}
  \end{minipage} %
  \begin{minipage}[b]{0.32\textwidth}
  \centering
    \includegraphics[width=\textwidth]{Figures/tmala_fm.png}
  \end{minipage}
   \caption{Comparison of \texttt{tULA}, \texttt{tULAc} and \texttt{tMALA} for the first moment evolving as a function of step size.}
\end{figure}

\begin{figure}
\centering
  \begin{minipage}[b]{0.85\textwidth}
  \centering
    \includegraphics[width=\textwidth]{Figures/doublewell_0_1_10_5samp_100dFirstMoment.png}
  \end{minipage}\\ %
  \begin{minipage}[b]{0.85\textwidth}
  \centering
    \includegraphics[width=\textwidth]{Figures/secondmoment_double_well_100d_10_5samp.png}
  \end{minipage} %
  \caption{}
  \label{fig:doubleWell_moment}
  \end{figure}
    
	\section{Beyond Moments}\label{sec:Beyond}
	While first and second moments give us some idea of the performance of our sampling algorithms, we ideally would like a fuller picture.  In this section we compare the performance of algorithms using the total variation distance, Wasserstein distance and Kullback--Leibler divergence.  Using these measures, we can compare the performance to theoretical upper bounds for \texttt{ULA}.

\subsection{Statistical Distances}
Let $\mathcal{B}(\R^d)$ denote the Borel $\sigma$-algebra on $\R^d$. Let $P$ and $Q$ be probability measures on the space $(\R^d, \mathcal{B}(\R^d))$.  Then we define the total variation distance, Kullback--Leibler divergence and Wasserstein metric as follows:

\begin{defn}[Total Variation]
The total variation distance between two probability measures $P$ and $Q$ on $(\Omega, \mathcal{F})$ is defined as
$$
\norm{P - Q}_{TV} = \sup_{A \in \mathcal{F}} \abs{P(A) - Q(A)}.
$$
\end{defn}
In other words, total variation measures the greatest possible difference between the probability of an event according to $P$ and $Q$.
\begin{prop}
If the set $\Omega$ is countable then this is equivalent to half the $L^1$ norm.
$$
\norm{P - Q}_{TV} = \frac{1}{2} \norm{P-Q}_1 = \frac{1}{2} \sum_{\omega \in \Omega} \abs{P(\omega) - Q(\omega)}
$$
\end{prop}
\begin{proof}
Let $B = \{\omega: P(\omega) \geq Q(\omega)\}$ and let $A \in \mathcal{F}$ be any event.  Then
$$
P(A) - Q(A) \leq P(A \cap B) - Q(A \cap B) \leq P(B) - Q(B).
$$
The first inequality holds since $P(\omega)-Q(\omega) < 0$ for any $\omega \in A \cap B^c$, and so the difference in probability cannot be greater if these elements are excluded.  For the second inequality, we observe that including further elements of $B$ cannot decrease the difference in probability.
Similarly,
$$
Q(A) - P(A) \leq Q(B^c) - P(B^c) = P(B) - Q(B)
$$
Thus, setting $A=B$, we have that $\abs{P(A)-Q(A)}$ is equal to the upper bound in the total variation distance.  Hence,
$$
\norm{P-Q}_{TV} = \frac{1}{2} \abs{P(B)-Q(B)+Q(B^c)-P(B^c)} = \frac{1}{2} \sum_{\omega \in \Omega} \abs{P(x)-Q(x)}
$$
\end{proof}

\begin{defn}[Kullback--Leibler Divergence]
Let $P$ and $Q$ be two probability measures on $(\Omega, \mathcal{F})$.  If $P \ll Q$, the Kullback--Leibler divergence of $P$ with respect to $Q$ is defined as
$$
D_{\text{KL}}(P\,||\,Q) = \int_\Omega \od{P}{Q} \log \left(  \od{P}{Q} \right) \dif  Q.
$$
\end{defn}
The Kullback--Leibler divergence from $Q$ to $P$ measures the information lost in using $Q$ to approximate $P$  \cite{anderson2004model} and is also known as the relative entropy.  It is worth noting that, unlike the other two measures considered here, the Kullback--Leibler divergence is not a metric, and in particular is not symmetric.

Finally we consider the Wasserstein distance.  If $P$ and $Q$ are probability measures on $(\Omega, \mathcal{F})$, we say that $\gamma$ is a transport plan between two probability measures $P$ and $Q$ if it is a probability measure on $(\Omega \times \Omega, \mathcal{F} \times \mathcal{F})$ such that for any Borel set $A \subset \mathcal{F}$, $\gamma(A \times \Omega)=P(A)$ and $\gamma(\Omega \times A) = Q(A)$.  We denote the set of all such transport plans by $\Pi(P,Q)$.  In simple terms, the set of transport plans, $\Pi(P,Q)$, represents the possible ways of transporting mass distributed according to $P$ to a distribution according to $Q$, without creating or destroying mass in the process.  The `effort' to transport mass is then represented by a cost function $d:\Omega \to \Omega$, so that $d(x,y)$ is the cost of moving unit mass from $x$ to $y$.

\begin{defn}[Wasserstein distance]
For two probability measures, $P$ and $Q$, the $p$-Wasserstein distance is given by
$$
W_p(P,Q) = \left( \inf_{\gamma \in \Pi(P,Q)} \int_{\Omega \times \Omega} d(x,y)^p d \gamma(x,y) \right)^{1/p}.
$$
\end{defn}

The Wasserstein distance represents the amount of `effort' required to move mass distributed according to $P$ to $Q$.  We restrict our attention to $L^1$-Wasserstein and $L^2$-Wasserstein distances, which is to say that we choose our cost function $d$ to be the Euclidean distance, and $p=1,2$.

One particular advantage of Wasserstein distance compared to total variation or Kullback--Leibler divergence is that bounds on Wasserstein distance can be used directly to bound the accuracy of the first and second moment approximations, and so for application to statistics there is some evidence to suggest it is the most appropriate of the three measures for our purposes \cite{dalalyan2019user}.

Due to impracticality of computing higher-dimensional Wasserstein distances, we use a computationally more feasible variant, the Sliced Wasserstein distance. First proposed in \cite{rabin2011wasserstein} and further elaborated on, for example, in \cite{gswd}, the Sliced Wasserstein distance exploits the fact that the Wasserstein distance between 1-dimensional probability measures $P, Q$ can be computed with an explicit formula $\abs{F^{-1}(t)-G^{-1}(t)}^p dt$ where $F$ and $G$ are the CDFs of $P$ and $Q$ respectively \cite{ramdas2017wasserstein}.


\begin{defn}[Sliced Wasserstein distance]
For two probability measures, $P$ and $Q$, the $L^p$ Sliced Wasserstein distance is given by
$$
SW_p(P,Q) = \left(\int_{\mathbb S^{d-1} }  W_p^p\left(\mathcal{RI}_P(\cdot, \theta), \mathcal{RI}_Q(\cdot, \theta) \right) d \theta \right)^{\frac 1 p}
$$
\end{defn}

where $\mathbb S^{d-1}$ is the $(d-1)$-dimensional sphere and $\mathcal RI$ denotes the Inverse Radon transform. In the above references, it is also proved that $SW_p$ is indeed a metric. The main reason why we can use the Sliced Wasserstein distance as an approximation to the Wasserstein distance is that these two metrics are equivalent\cite{Santa}.

\subsubsection{Numerical Comparison}
In this section we present a numerical comparison of the algorithms considered, extending the results of \cite{Brosse18tULA} which only considered first and second moments.  It should be noted that the results in the aforementioned paper are in dimension 100, while here our results are in dimension 2.  This is due to the bin-filling problem when approximating the density function, which is explained in section \ref{sec:Imp}.  We do not have evidence to suggest the performance of these algorithms would be similar in higher dimensions.



\subsection{Theoretical Non-asymptotic Error Bounds}

While the asymptotic behaviour of \texttt{ULA} and \texttt{MALA} is well-understood \cite{RT96}, the results do not consider the effect of dimension on the complexity of the algorithms.  For practical purposes, an understanding of the non-asymptotic behaviour of the algorithms would be useful and, in particular, theoretical results which could determine the step size and number of iterations required to guarantee an error of no more than some acceptable value $\epsilon$.

Theoretical bounds of this nature were first provided in \cite{dalalyan2017theoretical}, which proved bounds on the total variation distance between the distribution of the $n^{\text{th}}$ iterate of the unadjusted Langevin Algorithm, under restrictive assumptions, which will be outlined shortly.  In the case of a `warm start', where the distribution of the initial value is close to $\Pi$, it was shown that \texttt{ULA} had an upper bound of $\mathcal{O}(d/\epsilon)$ iterations to achieve precision level $\epsilon$.  This result was improved by \cite{durmus2016high, durmus2017nonasymptotic} which extended the analysis to the Wasserstein distance and dispensed with the assumption of a warm start, showing that the upper bound on iterations could be reduced to $\mathcal O (d/\epsilon)$, provided the Hessian of the potential is Lipschitz continuous.  Most recently, \cite{dalalyan2019user} has provided `user-friendly' bounds on the Wasserstein distance, further improving the constants in these bounds.

It is important to note here that these results provide explicit constants for the non-asymptotic behaviour for \texttt{ULA}.  This is in contrast to the theory for \texttt{MALA} \cite{bou2013nonasymptotic}, and the tamed \cite{Brosse18tULA} and higher order algorithms \cite{Sabanis18tHOLA}, for which only the existence of a constant is proven.  For the Leimkuhler--Matthews method, no such guarantees are known.  As such the non-asymptotic theory for the unadjusted algorithm is practically much more useful.

\subsubsection{Non-asymptotic Bounds for \texttt{ULA}}
We present without proof the results of \cite{durmus2017nonasymptotic, dalalyan2019user} which to our knowledge are the best known bounds in total variation and Wasserstein distance respectively. For both results we assume that the potential $U$ is continuously differentiable on $\R^d$, and there exist positive constants $m$ and $M$ such that $U$ is $m$-strongly convex and $M$-gradient Lipschitz, i.e. for all $x$, $y \in \R^d$,
\begin{align} \label{m-convex}
    &\text{($m$-strongly convex) \ } U(x) - U(y) - \grad U(y)^\top (x - y) \geq \frac{m}{2}\norm{x-y}_2^2\\
    &\text{($M$-gradient Lipschitz) \ } \norm{\grad U(x) - \grad U(y)}_2 \leq M \norm{x-y}_2. \label{M-GradLipschitz}
\end{align}

Denote the unique minimiser of $U$ by $y=\arg \min_{x \in \R^d} U(x)$.  Let $\nu_N$ denote the distribution of the $N^{\text{th}}$ sample

\begin{theorem}[Total Variation part (i)]
    Assume $h \in (0, 2/(m+M))$ and $U$ satisfies the assumptions (\ref{m-convex}, \ref{M-GradLipschitz}) above.  Let $\kappa = \frac{2mM}{m+M}$  Then for any initial value $x_0 \in \R^d$ and $N \geq 1$,
    $$
    \norm{\pi_h - \nu_N}_{TV} \leq \left\{ 4 \pi \kappa (1-(1-\kappa h)^{N/2} \right\}^{-1/2} (1-\kappa \gamma)^{N/2} \left\{ \norm{x_0 - y}_2 + (2 \kappa^{-1}d)^{1/2} \right\}
    $$
\end{theorem}

\begin{theorem}[Total Variation part (ii)]
    Assume $h \in (0, 1/(m+M))$, $U$ satisfies the assumptions (\ref{m-convex}, \ref{M-GradLipschitz}) above, and further, assume that $U$ is three times continuously differentiable, and there exists $L$ such that for all $x,y \in \R^d$,
    $$
    \norm{\grad^2 U(x) - \grad^2 U(y)} \leq L \norm{x-y}.
    $$
    Then
    \begin{align*}
        \norm{\pi_h - \pi}_{TV} &\leq (4 \pi)^{-1/2} \left\{ h^2 E_1(h, d) + 2dh^2 E_2(h)/(\kappa m)\right\}^{1/2}\\
        &+ (4 \pi)^{-1/2} \lceil log(h^{-1}/log(2) \rceil \left\{ h^2 E_1(h,d) + h^2 E_2(h)(2 \kappa^{-1}d + d/m) \right\}^{1/2}\\
        &+ 2^{-3/2} M \left\{ 2d h^3 L^2/(3\kappa) + dh^2\right\}^{1/2}
    \end{align*}
    where $E_1(h,d)$ and $E_2(h)$ are defined as
    \begin{align*}
        E_1(h,d) &= 2 d \kappa^{-1} \left\{2L^2 + 4 \kappa^{-1} (dL^2/3 + h M^4 /4) + h^2 M^4/6 \right\}\\
        E_2(h) &= L^4(4\kappa^{-1}/3 + h)
    \end{align*}
\end{theorem}

The triangle inequality gives our desired bound on $\norm{\pi - \nu_N}_{TV}$.  We next present the `user-friendly' bound in Wasserstein distance.

\begin{theorem}[Wasserstein distance]
    Assume $h \in (0, 2/M)$ and $U$ satisfies the assumptions (\ref{m-convex}, \ref{M-GradLipschitz}) above. 
    \begin{itemize}
        \item If $h \leq \frac{2}{m+M}$, then $W_2(\nu_N, \pi) \leq (1-mh)^N W_2(\nu_0, \pi) + 1.65 \frac{M}{m}(hp)^{1/2}.$
        \item If $h \geq \frac{2}{m+M}$, then $W_2(\nu_N, \pi) \leq (Mh-1)^N W_2(\nu_0, \pi) + \frac{1.65Mh}{2-Mh}(hp)^{1/2}.$
\end{itemize}
\end{theorem}
\begin{prop}
    If the initial value $X_0 = x_0$ is deterministic then,
    $$
    W_2(\nu_0, \pi)^2 = \int_{\R^p} \norm{x_0-x}_2^2 \pi(dx) \leq \norm{x_0-y}_2^2 + \frac{p}{m}.
    $$
\end{prop}

\begin{remark}
    If we choose $h$ and $N$ such that
    $$
    h \leq \frac{2}{m+M}, \quad e^{-mhN}W_2(\nu_0, \pi) \leq \epsilon/2, \quad 1.65\frac{M}{m}(hp)^{1/2} \leq \epsilon/2
    $$
    then $W_2(\nu_N,\pi) \leq \epsilon$.  Hence, for a deterministic initial value $X_0=x_0$, it is sufficient to choose
    $$
    h \leq \frac{m^2 \epsilon^2}{11M^2p} \wedge \frac{2}{m+M}, \quad hN \geq \frac{1}{m} \log \left( \frac{2(\norm{x_0-y}_2^2+p/m)^{1/2}}{\epsilon} \right)
    $$
    for a precision $\epsilon$ in $W_2(\nu_K, \pi)$.
\end{remark}

\begin{note}
    In practice, $\norm{x_0-y}_2$ may be difficult to calculate.  An alternative bound can easily be derived from the strong convexity of $U$ and the fact that $y$ minimises $U$:
    \begin{align*}
        m W_2(\nu_0, \pi)^2 &\leq m \norm{x_0-y}_2^2 + p\\
        &\leq 2(f(x_0) - f(y) - \nabla U(y)^{\top}(x_0-y) + p\\
        &= 2(f(x_0)-f(y))+p.
    \end{align*}
    If $U$ is bounded below by a constant, say $U \geq 0$, this provides an easily computable upper bound on $W_2(\nu_0, \pi)$.
\end{note}


	
	
	\section{Implementation} \label{sec:Imp}
	\input{WriteUp/Implementation}
    
 
    \section{Conclusion}
    In this report we have introduced Markov Chain Monte Carlo algorithms, with a particular focus on Langevin Monte Carlo methods. We then tested the performance of the algorithms based on a number of metrics and a broad range of parameters. Our main contribution is extending the work of \cite{Brosse18tULA} and producing a \textsc{Python} package ready for use by other researchers to implement their own methods to be tested against the algorithms we have presented here. As well as implementing their own methods, it is also written in a way that allows easy extension to different potentials and distributions. As well as this, we have extended the visualisation library of Rogozhnikov \cite{rogozhnikov2016hmc} to include LMC methods. This allows students and those newly introduced to the field to get an intuitive look at how these algorithms work, and the differences between them. 

We have shown that taming is a viable method of preventing divergence of Langevin-based algorithms and gives insight into a distribution when Metropolised algorithms are unable to, particularly in ill-conditioned problems. It is interesting to note the efficiency of the \texttt{LM} algorithm in particular, despite its relative simplicity. It does however suffer from divergence problems as \texttt{ULA}.

There is no simple answer as to which algorithm is `best'. Depending on the application, and the prior information given on the distribution, one could make the case for any of the algorithms presented here. If the general shape of the distribution is known and computational power is not a restriction, a simple Random Walk Metropolis algorithm may well be the best option, even in the superlinear case.

If very little is known about maxima of the distribution, the taming method is the best choice. It is able to quickly find the modes of the distribution without divergence, however once at the mode it often over estimates the width of the potential well.  This has provoked some investigation into `switching' methods, where the chain is initially started using \texttt{tULA} for a fixed time to find the mode, before switching to the \texttt{RWM} algorithm to better explore the well. This is the rationale behind the \texttt{HPH} algorithm in the package, however it has been omitted from the report due to a lack of theoretical justification. It may well turn out to be similar to existing adaptive time stepping methods, or tempering methods.

Higher order methods seem to come at too great a computational cost, especially in very high dimension, although the theory supporting them suggests that more work is to be done on the numerical implementation of such methods.

If this report has highlighted anything, it is that Metropolisation is not the final word in Langevin Monte Carlo, and other methods of approximating Langevin dynamics should be exploited and explored. Non-asymptotic bounds are of great importance in this area, as it is known that \texttt{MALA} converges in the limit but this is of little practical use. 

\subsection{Future Work}\label{subsec:future}
There are many possible ways in which the work presented here could be extended, both from a research and personal perspective. First, there is great scope for the improvement of our program. It is possible to add new analytic distributions to sample from, including those with non-smooth potentials as in \cite{durmus2018efficient}. Another obvious avenue would be to apply all the methods and analysis here to real (large) datasets, when the gradient is not known analytically. This would give the end user a much clearer impression of which algorithm to use in their given case. It would also slow down all gradient based methods as an unbiased estimator for the gradient would have to be calculated at every iteration. It may also be possible to speed up the higher order methods (\texttt{HOLA,tHOLA,tHOLAc}) by implementing a parallelised version, breaking up the complex iteration into smaller easier to manage sections. This in general is highly non-trivial due to the inherent dependence on the previous step in a Markov chain. Another important consideration is that of burn-in time. For the majority of our tests, we started from a minimiser of the potential well however our code is certainly not limited to this case. Doing so effectively removes any burn-in time as the chains are started in approximate stationarity. When started far away from the mode of a distribution, initial tests suggest taming is a highly preferable method to Metropolised algorithms. They take many more steps towards the mode, while Metropolised algorithms waste lots of time rejecting moves. In practice, this would greatly reduce the burn-in time -- a key feature of an effective MCMC method. 

It is also important that we test the accuracy of our measures. This is difficult, especially in high dimensions as no methods exist for numerically calculating the 2-Wasserstein metric that many of the theoretical bounds use. Furthermore, using kernel density estimation or histograms with few bins introduces an error that is difficult to quantify and reduce. Here we have only qualitative comparisons between metrics. This is a well known problem in numerical optimal transport and inherent in dealing with high dimensional datasets and problems.

Many other methods exist in MCMC that remain to be tested against the tamed algorithms, or incorporated in to our program. These include Hamiltonian Monte Carlo (HMC), manifold MALA (mMALA), underdamped Langevin Monte Carlo and stochastic gradient Langevin dynamics (SGLD) \cite{betancourt2017conceptual, Girolami2011,cheng2018,pitfalls}. For dealing with stiff problems, specific methods also exist that could be used as benchmarks for taming algorithms \cite{abdulle2013weak}. In the next section SGLD will be expanded on due to its popularity in machine learning and high dimensional problems. 

For the wider field, it is important that either `user-friendly' bounds on nonasymptotic error are developed in terms of a numerically implementable metric, or that methods are developed to accurately calculate the 2-Wasserstein metric. Even this will not solve the problem of approximating a high dimensional distribution using samples from a distribution; it is simply infeasible to be able to generate enough samples to get a good representation. 


     
    \subsection{Stochastic Gradient Langevin Dynamics} \label{subsec:SGLD}
    % \documentclass[a4paper]{article}

% \usepackage[english]{babel}
% \usepackage[utf8]{inputenc}
% \usepackage{amsmath}
% \usepackage{graphicx}
% \usepackage[colorinlistoftodos]{todonotes}
% \usepackage{titling}
% \usepackage[margin=40mm]{geometry}
% \usepackage{tikz}
% \usepackage{lipsum}
% \usepackage{amsfonts}
% \usepackage{amsmath,amssymb}
% \usepackage{breqn}

% \newtheorem{theorem}{Theorem}


% \newtheorem{lemma}[theorem]{Lemma}
% \newtheorem{corollary}{Corollary}[theorem]



% \begin{document}
% \setlength{\droptitle}{-0.5in}


\normalsize
\textbf{Stochastic Gradient Langevin Dynamics}\\
In this section, we will closely follow\cite{pitfalls}



\subsection{Introduction}

Normally, samples in machine learning are of huge sample sizes, for which most MCMC algorithms are not designed to process. 
As a result of the computational cost, several new approaches were proposed recently, Stochastic Gradient Langevin Dynamics (SGLD) is a popular one. 
SGLD is based on the Langevin Monte Carlo (LMC)
LMC – a discretization of a continuous-time process, it requires to compute the gradient of the log-posterior at the current fit of the parameter and avoid the accept/reject step.
SGLD – use unbiased estimator of the gradient of log-posterior based on subsampling, suitable for samples of huge size. 


\subsection{Governing Equation}

Langevin Stochastic Differential Equation (SDE):$$d\theta_t = -\nabla U(\theta_t)d +\sqrt{2}d B_t$$ where $(B_t)_{t \geq 0}$ is a d-dimensional Brownian motion. 

Euler discretization of the Langevin SDE:
$$\theta_{k+1} = \theta_k - \gamma \nabla U(\theta_k)+\sqrt{2\gamma} Z_{k+1}$$, where $\gamma > 0$ is a constant step size and $(Z_k)_{k\geq1}$ is a sequence of i.i.d standard d - dimensional Guassian vectors. 

To reduce the costs of the algorithms, we will switch to SGLD, for which we will replace $\nabla U$ with an unbiased estimate $\nabla U_0+(\frac{N}{p}) \sum_{i\in S}\nabla U_i$, where S is a minibatch of {1,...., N} with replacement of size p. Our iterations were then updated as
$$\theta_{k+1} = \theta_{k}-\gamma \Bigg (\nabla U_0(\theta_k)+\frac{N}{p}\sum_{i\in S_{k+1}}\nabla U_i(\theta_k)\Bigg )+\sqrt{2\gamma}Z_{k+1}$$

Stochastic Gradient Descent (SGD) is characterised by the same recursion as SGLD without the Gaussian noise, (the last term):


$$\theta_{k+1} = \theta_{k}-\gamma \Bigg (\nabla U_0(\theta_k)+\frac{N}{p}\sum_{i\in S_{k+1}}\nabla U_i(\theta_k)\Bigg)$$

\subsection{Analysis in Wasserstein distance}
\subsection{Definitions and Notations in Markov chain theory}
$\mathcal{P}_2(\mathbb{R}^d)$ the set of probablity measures with finite second momet.\\
$\mathcal{B}(\mathbb{R}^d)$ the Borel $\sigma$ - algebra of $\mathbb{R}^d$.\\
For $\lambda, \nu \in \mathcal{P}_2(\mathbb{R}^d)$, we define the Wasserstein distance by 

$$W_2(\lambda, \nu) =\inf_{\xi \in \Pi(\lambda, \nu)}(\int_{\mathbb{\mathbb{R}^d \times \mathbb{R}^d}}||\theta-\vartheta)||^2 \xi(d\theta, d\vartheta))^{\frac{1}{2}}$$
where, $\Pi(\lambda, \nu)$ is the set of probablity measures $\xi$ on $\mathcal{B}(\mathbb{R}^d)\otimes\mathcal{B}(\mathbb{R}^d)$ satisfying for all $A \in \mathcal{B}(\mathbb{R}^d), \xi(A \times \mathbb{R}^d)= \lambda(A)$ and $\xi (\mathbb{R}^d \times A) = \nu(A)$.\\
For any probablity measure $\lambda$ on $\mathcal{B}(\mathbb{R}^d)$, we define $\lambda R$ for all $A \in \mathcal{B}(\mathbb{R}^d)$ by $\lambda R(A) = \int_{\mathbb{R}^d}\lambda(d\theta)R(\theta, A)$.\\
For all $k\in \mathbb{N}*$, we define the Markov kernel $R^k$ recursively by $R^1 = R$ and for all $\theta \in \mathbb{R}^d$ and $A \in \mathcal{B}(\mathbb{R}^d)$, $R^{k+1}(\theta, A) = \int_{\mathbb{R}^d}  R^k(\theta, d\vartheta)R(\vartheta, A).$\\
A probablity measure $\bar{\pi}$ is invariant for R if $\bar{\pi}R = \bar{\pi}$.\\
Our algorithms LMC, SGLD, SGD and SGLDFP algorithms are homogeneous Markov chains with Markov kernels denoted $R_{LMC}, R_{SGLD}, R_{SGD}$ and $R_{FP}$.

\subsection{Results}
For lemma 1, Theorem 2 and Corollary 3, we assume H1, H2 and H3.
\begin{lemma}
For any step size $\gamma \in (0, \frac{2}{L})$, $R_{SGLD}$(respectively $R_{LMC}, R_{SGD}, R_{FP}$) has a unique invariant measure $\pi_{SGLD}\in \mathcal{P}_2(\mathbb{R}^d)$(respectively $\pi_{LMC}, \pi_{SGD}, \pi_{FP}$). In addition, for all $\gamma \in (0, \frac{1}{L}], \theta\in \mathbb{R}^d and k\in\mathbb{N}$,
$$W_2^2(R_{SGLD}^k(\theta, \cdot), \pi_{SGLD})\leq(1-m\gamma)^k\int_{\mathbb{R}^d}||\theta-\vartheta||^2\pi_{SGLD}(d\vartheta)$$
same inequality holds for LMC, SGD and SGLDFP.
\end{lemma}
\begin{theorem}
For all $\gamma\in(0,\frac{1}{L}], \lambda, \nu\in \mathcal{P}_2(\mathbb{R}^d) and n\in\mathbb{N}$, we have the following upper- bounds in Wasserstein distance between
\begin{enumerate}
	\item 
	LMC and SGLDFP,

\begin{dmath}	
W_2^2(\lambda R_{LMC}^n, \nu R_{FP}^n)\leq(1-m\gamma)^nW_2^2(\lambda, \nu) + \frac{2L^2\gamma d}{pm^2}+\frac{L^2\gamma^2}{p}n(1-m\gamma)^{n-1}\int_{\mathbb{R}^d}||\vartheta-\theta*||^2 \mu(d\vartheta)
\end{dmath},
	\item 
	the Langevin diffusion and LMC,
\begin{dmath}
W_2^2(\lambda R_{LMC}^n, \mu P_{n\gamma})\leq2(1-\frac{mL\gamma}{m+L})^nW_2^2(\lambda, \mu)+d\gamma\frac{m+L}{2m}(3+\frac{L}{m})(\frac{13}{6}+\frac{L}{m})\\+ne^{-(\frac{m}{2})\gamma(n-1)}L^3\gamma^3(1+\frac{m+L}{2m})\int_{\mathbb{R}^d}||\vartheta - \theta*||^2 \mu(d\vartheta)
\end{dmath},
	\item 
	SGLD and SGD
	\begin{dmath}
	W_2^2(\lambda R_{SGLD}^n, \mu R_{SGD}^n)\leq (1-m\gamma)^n W_2^2(\lambda, \mu)+\frac{(2d)}{m}.
	\end{dmath}
\end{enumerate}
\end{theorem}
Proof omitted.
\begin{cor}
Set $\gamma - \frac{\eta}{N} with \eta \in (0, \frac{1}{(2L)}]$ and assume that $lim \inf_{N \to \infty}mN^{-1}>0$. Then
\begin{enumerate}
\item
for all $n \in N$, we get $W_2(R_{LMC}^n(\theta*, \cdot), R_{FP}^{n}(\theta*, \cdot)) = \sqrt{d\eta}\mathcal{O}(N^{-\frac{1}{2}})$ and $W_2(\pi_{LMC}, \pi_{FP}) = \sqrt{d\eta}\mathcal{O}(N^{-\frac{1}{2}})$.
\item 
for all $n\in \mathbb{N}, we get W_2(R_{SGLD}^{n}(\theta*, \cdot), R^n_{SGD}(\theta*, \cdot)) = \sqrt{d}\mathcal{O}(N^{-\frac{1}{2}})$, and $W_2(\pi_{SGLD}, \pi_{SGD}) = \sqrt{d}\mathcal{O}(N^{-\frac{1}{2}})$.
\end{enumerate}
\end{cor}



    \section{Acknowledgements}
    The authors were supported by The Maxwell Institute Graduate School in Analysis and its Applications, a Centre for Doctoral Training funded by the UK Engineering and Physical Sciences Research Council (grant EP/L016508/01), the Scottish Funding Council, Heriot-Watt University and the University of Edinburgh
    
	\bibliography{langevinMC.bib}
	\bibliographystyle{plain}

\end{document}
